% In this file you should put the actual content of the blueprint.
% It will be used both by the web and the print version.
% It should *not* include the \begin{document}
%
% If you want to split the blueprint content into several files then
% the current file can be a simple sequence of \input. Otherwise It
% can start with a \section or \chapter for instance.

\chapter{Main results}

There are our main results.

\begin{definition}[Weighted triangulation]
  \label{Weighted-triangulation}
  A weighted triangulation $(T, \Theta)$ is a triangulation equipped with a weight $\Theta : E → [0, \pi/2]$.
\end{definition}

\begin{definition}[Circle packing Metric, piecewise linear length, Cone angle]
  \label{Metric-Length-ConeAngle}\uses{Weighted-triangulation}
  We call $r=\left (r_{v}\right)_{v \in V} \in \mathbb{R}_{>0}^{V}$ a metric on ( $\left. T, \Theta\right)$ and define the length $\ell (e ; r)$ of an edge $e$ with endpoints $v, u$ with respect to $r$ by
  $$
  \ell(e ; r):=\binom{\text { the distance between the centers of two circles on the universal }}{\text { cover of } S \text { with radii } r_{v}, r_{u} \text { intersecting at angle } \Theta(e)},
  $$
  Which determines the angle at each vertex in triangles of $(T, \Theta)$. The cone angle $a_{v}(r)$ at the vertex $v$ with respect to $r$ is the sum of each angle at $v$ in all triangles of $(T, \Theta)$ having $v$ as one of the vertices.
\end{definition}

\begin{definition}[combinatorial curvature]
  \label{combinatorial-curvature}\uses{Metric-Length-ConeAngle}
  We call $K_v (r) := 2π − a_v (r)$ the curvature at the vertex $v$ with respect to $r$
\end{definition}

\begin{definition}[circle pattern metric]
  \label{circle-pattern-metric}\uses{combinatorial-curvature}
  A metric is called a circle pattern metric provided its curvature is identically zero at each vertex.
\end{definition}

\begin{definition}[combinatorial Ricci flow]
  \label{combinatorial-ricci-flow}\uses{Metric-Length-ConeAngle, combinatorial-curvature}
  the combinatorial Ricci flow, which is a family of metrics $\{r (t)\}_{t \in[0, T)}$ that satisfies
  $$
  \frac{d}{d t} r_{v}(t)=-K_{v}(r(t)) \sigma_{S}\left(r_{v}(t)\right), \quad \sigma_{S}\left(r_{v}\right):=\left\{\begin{array}{ll}
  r_{v} & \text { if } \chi(S)=0 \\
  \sinh r_{v} & \text { if } \chi(S)<0
  \end{array}\right.
  $$
\end{definition}

\begin{theorem}[Thurston-Chow-Luo theorem]
  \label{thurston-chow-luo}
  Let $(T, \Theta)$ be a weighted triangulation of a surface $S$ of nonpositive Euler characteristic. The following conditions (I), (II), and (III) are equivalent to each other.

  (I) There exists a unique circle pattern metric up to a scalar multiple if $\chi (S)=0$

  (II) It holds for any nonempty proper subset $U$ of $V$ that
  $$
  \phi(U):=-\sum_{f \in \operatorname{Lk}(U)}\left(\pi-\Theta\left(e_{v(f)}^{f}\right)\right)+2 \pi \chi\left(\tau_{U}\right)<0
  $$
  Where $\tau_{U}$ is the $C W$ -subcomplex of $T$, consisting of all cells whose vertices are contained in $U$, and $f \in \operatorname{Lk}(U)$ is an element in $F$ such that one vertex $v (f)$ in $f$ belongs to $U$ and neither of the endpoints of the edge $e_{v (f)}^{f}$ in $f$ opposite $v (f)$ belongs to $U$.

  (III) Given any metric $r$ on $(T, \Theta)$, the combinatorial Ricci flow with initial data $r$ exists for all time and converges on $\mathbb{R}_{>0}^{V}$ at infinity.
\end{theorem}

\begin{definition}[phi def]
  \label{phi-def} \uses{thurston-chow-luo}
  For any nonempty proper subset $U$ of $V$ that
  $$
  \phi(U):=-\sum_{f \in \operatorname{Lk}(U)}\left(\pi-\Theta\left(e_{v(f)}^{f}\right)\right)+2 \pi \chi\left(\tau_{U}\right)<0
  $$
  Where $\tau_{U}$ is the $C W$ -subcomplex of $T$, consisting of all cells whose vertices are contained in $U$, and $f \in \operatorname{Lk}(U)$ is an element in $F$ such that one vertex $v (f)$ in $f$ belongs to $U$ and neither of the endpoints of the edge $e_{v (f)}^{f}$ in $f$ opposite $v (f)$ belongs to $U$.
\end{definition}

\begin{theorem}[Main Theorem]
  \label{main-theorem} \uses{combinatorial-ricci-flow, phi-def}
  Let $(T, \Theta)$ be a weighted triangulation of a surface $S$ of nonpositive Euler characteristic such that $\phi (U) \leq 0$ holds for any $U \subset V$ and $Z_{T}:=\{z \in V \mid$ there exists a proper subset $Z$ of $V$ such that $z \in Z$ and $\phi (Z)=0\}$ is nonempty. Then for any metric $r$ on $(T, \Theta)$, the combinatorial Ricci flow $\{r (t)\}_{t \geq 0}$ with initial data $r$ does not converge on $\mathbb{R}_{>0}^{V}$ at infinity. However
  $$
  \lim _{t \rightarrow \infty} K_{v}(r(t))=0
  $$
  Holds for any vertex $v$.

  On the one hand, for $\chi (S)=0,\{r (t)\}_{t \geq 0}$ does not converge on $\mathbb{R}_{\geq 0}^{V}$ at infinity. However if we fix an arbitrary $v \in V \backslash Z_{T}$, then the limit
  $$
  \rho_{u}:=\lim _{t \rightarrow \infty} \frac{r_{u}(t)}{r_{v}(t)}
  $$
  Exists for any $u \in V$, where $Z_{T}=\left\{z \in V \mid \rho_{z}=0\right\}$ holds and $\left (\rho_{u}\right)_{u \in V \backslash Z_{T}}$ is a unique circle pattern metric with normalization $\rho_{v}=1$ on a certain weighted triangulation with vertices $V \backslash Z_{T}$.

  On the other hand, for $\chi (S)<0,\{r (t)\}_{t \geq 0}$ converges on $\mathbb{R}_{\geq 0}^{V}$ at infinity, where we have $Z_{T}=\left\{z \in V \mid \lim _{t \rightarrow \infty} r_{z}(t)=0\right\}$ holds and the limit of $\left (r_{v}(t)\right)_{v \in V \backslash Z_{T}}$ at infinity is a unique circle pattern metric on a certain weighted triangulation with vertices $V \backslash Z_{T}$.
\end{theorem}
